\documentclass[options]{article}
\begin{document}
<<<<<<< HEAD
\textbf{CLASSICAL SIMULATION OF QUANTUM SYSTEMS}

\section{\textbf{ Introduction}}
Quantum simulation permits the study of quantum systems that are difficult to study in the laboratory and impossible to study with supercomputers.
Richard Feynman suggested that it takes a quantum computer to simulate large quantum systems, but a new study shows that a classical computer can work when the system has loss and 


\section{\textbf{ Background }}
Only very few problems in physics can be solved exactly or even in closed form. In the vast majority of cases, more or less sophisticated numerical techniques are needed. As the computational effort may be large, the number and kinds of problems accessible to computational physics increases steadily with the ever increasing power of computers. Many tasks scale in a characteristic way with the problem size. The canonical example is a system of N interacting, point-like, classical particles: this system has a phase space of dimension 6N, and tracking the time evolution caused by the Hamiltonian, i.e., following a trajectory in phase space, means working with an amount of data linear in the system size. The computational effort to describe generic classical systems typically scales as a low polynomial in N. Thus, simulations of many-body systems are now possible for quite large particle numbers N. Nevertheless, classical many-body physics is far from being fully explored.

\section{\textbf{ Problem Statement}}
If quantum systems are used as the basic carrier of information, an advantage can be gained rather than using classical variables. For instance, with quantum systems fewer resources are required

\section{\textbf{ Aim and Objectives}}

\subsection{Main objectives}
Feynman stated that calculating properties of arbitrary quantum model on a classical device is a seemingly very inefficient thing to do but a quantum system might be able to do this efficiently taking a time that scales at most polynomially with the particle number.The aim of this project is to compare the resources required by the classical simulation with those required when quantum systems are used.

\subsection{Specific Objectives}
The project will investigate the classical simulation of quantum systems in simple scenarios in which quantum systems are communicated from one party to another. 

The project will investigate the classical simulation of quantum systems in simple scenarios in which quantum systems are measured to produce correlated outcomes. 

\section {\textbf{Scope}}

\section{\textbf{Literature Review }}


\section {\textbf{Methodology}}

\section{\textbf{References}}
R.P.Feynman, “Simulating physics with computers,”

https://physics.aps.org/articles/v9/66 

W.P.schleich,Quantum Optics in Phase Space 



=======
>>>>>>> parent of 5642776... latex








\end{document}
