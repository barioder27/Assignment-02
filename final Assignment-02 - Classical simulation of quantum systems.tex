\documentclass[options]{article}
\begin{document}
\begin{center}
\textbf{CLASSICAL SIMULATION OF QUANTUM SYSTEMS}

\textbf{Group Members}

MULUMBA MIYINZA SAMSON 15/U/5678/EVE

MUWANGUZI BARABARA 15/U/20482/EVE

MUGISHA WILLIAM 15/U/7917/EVE

BARIYO DERRICK   15/U/4401/EVE




\end{center}

\section{\textbf{ Introduction}}
Quantum simulation permits the study of quantum systems that are difficult to study in the laboratory and impossible to study with supercomputers.
The field of quantum computing originated with a question posed by Richard Feynman. He asked whether or not it was feasible to simulate the behavior of quantum systems using a classical computer, suggesting that a quantum computer would be required instead but a new study shows that a classical computer can work when the system has loss and noise.


\section{\textbf{ Background }}
Only very few problems in physics can be solved exactly or even in closed form. In the vast majority of cases, more or less sophisticated numerical techniques are needed. As the computational effort may be large, the number and kinds of problems accessible to computational physics increases steadily with the ever increasing power of computers. Many tasks scale in a characteristic way with the problem size. The canonical example is a system of N interacting, point-like, classical particles: this system has a phase space of dimension 6N, and tracking the time evolution caused by the Hamiltonian, i.e., following a trajectory in phase space, means working with an amount of data linear in the system size. The computational effort to describe generic classical systems typically scales as a low polynomial in N. Thus, simulations of many-body systems are now possible for quite large particle numbers N. Nevertheless, classical many-body physics is far from being fully explored.

\section{\textbf{ Problem Statement}}
If quantum systems are used as the basic carrier of information, an advantage can be gained rather than using classical variables. For instance, with quantum systems fewer resources are required

\section{\textbf{ Aim and Objectives}}

\subsection{Main objectives}
Feynman stated that calculating properties of arbitrary quantum model on a classical device is a seemingly very inefficient thing to do but a quantum system might be able to do this efficiently taking a time that scales at most polynomially with the particle number.The aim of this project is to compare the resources required by the classical simulation with those required when quantum systems are used.

\subsection{Specific Objectives}
\begin{itemize}
   \item  To investigate the classical simulation of quantum systems in simple scenarios in which quantum systems are communicated from one party to another.
   \item  To investigate the classical simulation of quantum systems in simple scenarios or in which quantum systems are measured to produce correlated outcomes. 
   \item To compare the resources required by the classical simulation with those required when quantum systems are used.
\end{itemize}

\section {\textbf{Scope}}
This research looks at the efficient simulation of quantum systems that have got sufficiently large loss and noise. 
This follows a study by Saleh Rahimi-Keshari from the University of Queensland, Australia that demonstrated that a quantum process that was believed to require an exponentially large number of steps to simulate on a classical computer could in fact be simulated in an efficient way if the system in which the process occurs has sufficiently large loss and noise.


\section{\textbf{Literature Review }}
Simulating quantum mechanics is known to be a difficult computational problem, especially when dealing with large systems. However, this difficulty may be overcome by using some controllable quantum system to study another less controllable or accessible quantum system, i.e., quantum simulation. 
Quantum simulation promises to have applications in the study of many problems in, e.g., condensed-matter physics, high-energy physics, atomic physics, quantum chemistry, and cosmology. Quantum simulation could be implemented using quantum computers, but also with simpler, analog devices that would require less control, and therefore, would be easier to construct. 

A number of quantum systems such as neutral atoms, ions, polar molecules, electrons in semiconductors, superconducting circuits, nuclear spins, and photons have been proposed as quantum simulators. This review outlines the main theoretical and experimental aspects of quantum simulation and emphasizes some of the challenges and promises of this fast-growing field.


\section {\textbf{Methodology}}
There are quite a number of methods, approaches and technics that can be used to classify the quantum systems which include:
\begin{enumerate}
  \item \textbf{Nonperturbative renormalization group transformation (NRG),}
Which give some guidelines for calculating physical quantities, and various applications which include variants of the original Kondo problem such as the non-Fermi-liquid behavior in the two-channel Kondo model, dissipative quantum systems such as the spin-boson model, and lattice systems in the framework of the dynamical mean-field theory.

   \item \textbf{Quantum Spectral Transform Method (QSTM), }
Developed as a result of a synthesis of two major directions in the modern theory of exactly soluble systems. The first of them is based on the tradition of studying the exactly soluble models of solid states. This method integrates to many others like the Classical Spectral Transform Method (CSTM), having yielded a lot of important results. The QSTM yields benefits giving solutions to:
\begin{itemize}
   \item  The sine—Gordon model 13, 20, 26 i.e. finding its mass spectrum and S—matrix.
   \item  The solution of the quantum inverse problem for the nonlinear.
\end{itemize}

   \item \textbf{Matrix product states and projected entangled pair states (MPS).}
This method analyses systems with unique as well as degenerate ground states and in both the absence and the presence of symmetries. These symmetries consist of two parts, one of which acts by permuting the ground states, while the other acts on individual ground states, and phases are labeled by both the permutation action of the former and the cohomology class of the latter.

\end{enumerate}

\section{\textbf{References}}
R.P.Feynman, “Simulating physics with computers,”

https://physics.aps.org/articles/v9/66 

W.P.schleich,Quantum Optics in Phase Space 







\end{document}
